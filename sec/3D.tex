% !TEX root = ../noninv.tex

\section{Examples in 3 dimensions and Turaev-Viro theories}\label{sec:3D}

LUKAS

In this section we will explain how the higher categorical perspective on state sum models proposed here can be used recover the well known Turaev-Viro model. The proof that the Street cone constructed in this section defines state sum data relies on the connection to the orbifold construction which we discuss in the next section.

Before we can try to construct 3-dimensional state sum data we need to fix an appropriate pivotal 3 category $\cC$. For this we will use the Euler completion of the delooping $B\CYCat$ of semi-simple Calabi-Yau categories~\cite{3DOrb} which we define now:

\begin{definition}
	A \emph{semisimple Calabi--Yau category} is a semi-simple linear category $\Ca$ with finitely many simple objects together with a linear
	trace map $\operatorname{tr}_c \colon \End (c) \longrightarrow \C$ for each $c\in \Ca$ such that
	\begin{itemize}
		\item
		for all morphisms $f\colon c \longrightarrow c'$ and $g\colon c' \longrightarrow c$ in $\Ca$ the relation $\operatorname{tr}_{c'}(f\circ g)= \operatorname{tr}_c({g\circ f})$ holds, and
		\item
		for all $c,c'\in \Ca$ the pairing $\Ca(c,c') \otimes \Ca(c',c) \xrightarrow{\  \circ  \ } \Ca(c,c) \xrightarrow{\operatorname{tr}_c} \C$ is non-degenerate.
	\end{itemize}
We denote by $\CYCat$ the symmetric monoidal 2-category of semisimple Calabi--Yau categories, linear functors and natural transformations. The tensor product is the Delegine product which we denote by $\boxtimes$. 
\end{definition}
The trace is required to make $\CYCat$ a pivotal category, see~\cite[Proposition 5.6]{3DOrb}.  

The Euler completion $E(B\CYCat)$ is defined in~\cite[Section 5.1.3]{3DOrb} and effectively allows one to include certain normalisation factors into the state sum model we construct later. Its objects consist of a pair of an object in $B\CYCat$ and of which there is only one $*$ and an invertible endomorphism $\psi \in \operatorname{Aut} (\id_{\id_*})=\C^\times$. The composition of morphisms is not influenced by introducing these aditional objects. However, they change the pivotal structure such that powers of $\psi$ will be inserted in bubbles. We need this small adjustment to get the right normalization factors. For more details we refere to~\cite[Section 5.1.3]{3DOrb}.  

Let us now turn to oriented Street 3-cones with values in $E(B\CYCat)$. They consist of an inevitable complex number specifying the functor $\oriental_0 \to E(B\CYCat)$. A compatible functor $\oriental_1 \to E(B\CYCat)$ is completely detriment by specifying a Calabi-Yau category $\cC$. A functor out of $\oriental_2$ compatible with the choices so far corresponds to a functor $\otimes \colon \cC \boxtimes \cC \to \cC$. Finally there are two natural transformations $\alpha_+\colon \otimes \circ (\otimes \boxtimes \id ) \Longrightarrow \otimes \circ (\id \boxtimes \otimes )$ and $\alpha_-\colon \otimes \circ (\id \boxtimes \otimes ) \Longrightarrow \otimes \circ (\otimes \boxtimes \id )$ which correspond to the functors out of $\oriental_3^+$ and $\oriental_3^-$, respectively.    

Specifying this data allows us to define an invariant of triangulated 3-dimensional manifolds. Among the many conditions required for an oriented 3-Street cone to be triangulation invariant one finds that $\alpha_+$ and $\alpha_-$ has to satisfy the well known pentagon equation. A natural place to look for state sum data are monoidal categories, where $\alpha_-$ is just the inverse of $\alpha$.\footnote{However, note that, there are no relations enforcing them to be inverse to each other in general. There is a notion of special 3-cone which would include this condition. The moves one has to impose in this case can be found in~\cite[Section 3.4]{carqueville2016orbifoldcompletion}}  

The Turaev-Viro construction is based on a spherical fusion category
\begin{definition}
    spherical fusion category....
\end{definition}
\begin{example}
    ??
\end{example}
Let $\mathcal{S}$ be a spherical fusion category the trace defines a canonical Calabi-Yau structure on $\mathcal{S}$ and hence $\mathcal{S}$ is an associative algebra in $\CYCat$ which we can use to define state sum data in $\Ca$ by sending the unique point in $\oriental_0$ to $(*,\dim (S)^{-\tfrac{1}{2}})$, where $\dim S = ...$ is the dimension of $S$. The 1-morphism in $\oriental_1$ is send to $S$ with its Calabi-Yau structure coming from the spherical trace. The 2-morphism in $\oriental_2$ is send to the tensor product functor $\otimes \colon S\boxtimes S \to S$. The 3-morphism in $\oriental_3^+$ is send to the associator and finally the 3-morphism in $\oriental_3^-$ to its inverse.