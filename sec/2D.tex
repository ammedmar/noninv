\section{Examples in 2 dimensions} 

Let $\mathcal{C} := B\Vect$ denote the $2$-category that deloops the monoidal category of finite-dimensional vector spaces.
Explicitly, is has a single object $*$ with endomorphism category given by $\Vect$. 
Horizontal composition is given by the tensor product of vector spaces.
We spell out what are state sum models with values in $\mathcal{C}$.
Since $\mathcal{C}$ has a single object, we assign to vertices $* \in \mathcal{C}$.
To edges we assign a single vector space $A \in \End_\mathcal{C} *$.
Finally we have to assign linear maps $\mu, \Delta$ to the $2$-simplex with its two different orientations $\Delta^-_2, \Delta^+_2$.
Next we have to require compatibility of this assignment with the maps in the oriented version of Street's diagram:
\[
\begin{tikzcd}
\mathcal{O}(\Delta^+_2)    & &
\\
& \mathcal{O}(\Delta^1) \ar[lu, bend right = 25, shorten <=0.2cm, shorten >=0.2cm] \ar[lu, bend left = 25, shorten <=0.2cm, shorten >=0.2cm] \ar[lu] \ar[ld, bend right = 25, shorten <=0.2cm, shorten >=0.2cm] \ar[ld, bend left = 25, shorten <=0.2cm, shorten >=0.2cm] \ar[ld] & \mathcal{O}(\Delta^0) \ar[l, bend left=20] \ar[l, bend right=20]
\\
\mathcal{O}(\Delta^-_2) & &
\end{tikzcd}
\]
This will tell us that the linear maps assigned to the two $2$-simplices will have source and target $\mu: A \otimes A \to A, \Delta: A \to A \otimes A$.

Now recall that for a two-dimensional state-sum model, we require the condition that it extends to oriented simplices in dimension three:

\begin{lemma}
    The data $(A,\mu,\Delta)$ defines a two-dimensional state-sum model if and only if it is a (non-unital and non-counital) $\Delta$-separable\footnote{The notion of $\Delta$-separable is unrelated to the notion of a separable algebra.} symmetric Frobenius algebra, i.e.
    \begin{itemize}
        \item $\mu$ is associative
        \item $\Delta$ is coassociative
        \item $\Delta$-separable: $\mu \circ \Delta = \id_A$.
        \item Frobenius: $\Delta \circ \mu = (\id_A \otimes \mu) \circ (\Delta \otimes \id_A) = (\mu \otimes \id_A) \circ (\id_A \otimes \Delta)$
        \item symmetric:\textcolor{red}{What does symmetric mean non-unitally and how does it arise?}
    \end{itemize}
\end{lemma}
\LM{Not so sure about the if and only if}
\begin{proof}
    This result is closely related to \cite[Proposition 3.4]{carqueville2016orbifoldcompletion}....
    They are related by the oriented two-dimensional Pachner moves \cite[Example 3.4(i)]{carqueville2019orbifolds}...
    The 2-2 moves give the Frobenius and (co)associativity relations.
    The 1-3 moves give $\Delta$-separable and... \textcolor{red}{how to deal with no units?}
\end{proof}
