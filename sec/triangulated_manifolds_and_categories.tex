\section{From triangulated manifolds to $n$-categories}
\LM{Maybe the appendix should go here}

\subsection{Street's construction}

We use the recursive definition of $n$-category as a category enriched over $(n-1)$-categories, with $\omega$-categories defined as the limiting notion of these.
The starting point of our construction is \textit{Street's diagram}
\[
\begin{split}
	\cO \colon \simplex \, &\to \wCat \\
	[n] &\mapsto \, \cO_n,
\end{split}
\]
a functor from the simplex category to the category of $\omega$-categories \cite{street1987orientals}.
Please consult \cref{s:categories} for a more detailed explanation.
The \textit{free $\omega$-category} functor associates to any simplicial set $X$ the $\omega$-category
\[
\cO(X) = \colim_{\simplex^n \downarrow X} \, \cO_n.
\]
For any $\omega$-category $\cat{C}$, the collection of natural functors $\cO(X) \to \cC$ is in bijection with that of cones under $\cO$ with apex $\cat{C}$.\anibal{Maybe there is more structure than just sets.}
