% !TEX root = ../noninv.tex

\section{Path integral aka state sums}

In this section we outline the construction of state-sum invariants. We will only consider `local
action functionals'. In our combinatorial setting this means that the action is a sum over top dimensional
simplicies.
\begin{definition}
Let $\mathcal{C}$ be a fusion $n$-category
an exponentiated \emph{local $d$-dimensional action functional} is a functor $S\colon \mathcal{F}_\mathcal{C}(\Delta^d) \to \bC $ where
$\bC$ is the discrete category with objects real numbers modulo integers.  \LM{There should be some linearity...}
\end{definition}
Given a field configuration $f\in \mathcal{F}_\mathcal{C}(X)$ we define its action $S(f)$ as the product
over all top simplicies in the triangulation $X$ weighted with a sign corresponding to whether the branching
structure matches with the orientation of $M$ or not:
\begin{align}
S(X,f)= \prod ...
\end{align}
The \emph{partition function} of the non-invertible lattice gauge theory based on $S$ is the path integral
\begin{align}
\mathcal{Z}(X) = \int\displaylimits_{f\in \mathcal{F}_\mathcal{C}(X)} S(X,f)
\end{align}
where the path integral needs to be defined. We call a non-invertible lattice gauge theory \emph{topological}
if the resulting path integral only depends on the topology of $X$ and not the specific triangulation. \
In the case $\cat{C}$ is finite $n$-groupoid there exist an
essentially unique integration theory in terms of homotopy cardinality. To the best of our knowledge there
is no theory of the type of integration needed to define the partition function in general. The types
of sum appearing in three and four dimensional state sum models~\cite{} together with the case of higher groupoids  seem to be a good starting point for defining the relevant notions. In this paper we concretely define the
case $n=0$, i.e. the case where the non-invertible symmetry can be described by a finite dimensional complex
algebra $A$. In this case a field corresponds to a labelling of all edges of the triangulation with
elements of $A$. To define the path integral we pick a basis $\{ a_i \}$ of $A$ and sum over labelings
with elements of the basis. \LM{Is this independent of the choice of a basis?} weighted by a factor of $\tfrac{1}{\dim A } $ for every vertex of the triangulation. There is a natural guess on how to extend
this to higher fusion categories. Namely by summing over a basis of simple objects, simple 1-morphisims,...
weighted by their dimensions. However, we will leave this to further work and restrict to the most simple example here.

\begin{example}
\LM{2D-state sum}
\end{example}

\begin{example}
We explain how to reproduce the original Dijkgraaf-Witten theory in our framework.
So let $G$ be a finite group and $\tau \in Z^d(G, \R/ \Z)$ a closed $d$-cochain on the standard model for group cohomology.
In more abstract terms, it is a map $BG \to B^{d} \R/\Z$ of simplicial sets where $BG$ is seen as a simplicial set through the nerve construction.
The twist $\tau$ is what is typically called the level for Chern-Simons theory and will be the essential input ingredient to defining the action.
Let $X$ be a triangulation of our $d$-dimensional spacetime manifold $M$.
To define the necessary path integrals, we will need $X$ to be finite, which is always possible in case $M$ is compact.

Principal $G$-bundles over $M$ form the $1$-type
\[
\Map(X, BG) \cong \Pi_{\leq 1} \Map(X, BG)
\]
which in the current setting we can represent algebraically in a concrete way as follows.
Objects of this groupoids are gauge fields $A$, which are defined to be labeling $1$-simplices of $X$ by elements of $G$.
Indeed, a map $X$ to $BG$ is equivalent to a map from $1$-simplices of $X$ to $G$.
For example, to a $2$-simplex in $X$ with edges $e_1, e_2, e_3$ it assigns the $2$-simplex $(A(e_1), A(e_3)) \in G \times G$ in $BG$.
One can think of this $2$-simplex as having edges labeled by group elements $A(e_1), A(e_2), A(e_3)$ with the condition $A(e_2) = A(e_1) A(e_3)$.
For $G$ abelian, these are closed $1$-cocycles on $X$ with values in $G$.
Explicitly this combinatorially models a principal $G$-bundle on $M$ as follows....
Morphisms in the groupoid from $A_1$ to $A_2$ are gauge transformations $f$ between gauge fields.
These are defined as assignments of group elements to vertices such that
\[
f(v_0) A_2(e) = A_1(e) f(v_1)
\]
if $e$ is an edge from $v_0$ to $v_1$.
Such labelings indeed corresponds to a simplicial homotopy between the maps $A_1, A_2: X \to BG$.
In case $G$ is abelian, this is saying that $f$ is a $0$-cochain and $A_2 = df \cdot A_1$.

The action on a field $A$ is now abstractly given by pairing the top cochain defined by the composition
\[
X \xrightarrow{A} BG \xrightarrow{\tau} B^d \R / \Z
\]
with the fundamental class of $X$.
Note that to obtain this fundamental class we need to have an orientation on $M$.
Explicitly, the action is given by
\[
S(A) = \int_M A^*(\tau) = \frac{1}{|X_n|} \prod_{\sigma \in X_d} \tau(A(e_1), \dots, A(e_d))
\]
where $e_1, \dots, e_d$ are...
To connect this example to the main text, we take $n = 0$ in which case we have to pick a fusion $0$-category $\mathcal{C}$, which is a finite-dimensional semisimple algebra.
We take it to be the group algebra of $G$.
A lattice gauge field as in the definition above is then an arbitrary assignment of elements of the group algebra to edges of the triangulation.
Note that usually the fields of Dijkgraaf-Witten theory are taken to be principal $G$-bundles over $M$, so one might expect an extra cocycle condition on the assignment.
We instead enforce this condition by taking the exponentiated action to be identically equal to zero on field configurations for which the assignment does not define a map $X \to BG$ of simplicial sets.
\end{example}

\begin{example}
\LM{Something non-invertible in higher D which will be new. Maybe $\bC[N]$ for a monoid }
\end{example}