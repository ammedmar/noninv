% !TEX root = ../noninv.tex

\section{Categories of fields and actions}

LUUK

In traditional gauge theory, physicists consider the collection $\mathcal{F}$ of connections on a principal $G$-bundles $P \to M$ for a Lie group $G$.
Here $M^n$ is an $n$-dimensional spacetime manifold and connections are typically called gauge fields.
One then identifies connections $A_1, A_2 \in \mathcal{F}$ if they are related by a gauge transformation $g \in \mathcal{G}$, i.e. a bundle automorphism of $P$.
In quantum field theory, one is lead to ponder the meaning of integrals over the space of gauge fields modulo gauge transformations
\begin{equation}
    \label{eq:pathintegral}
Z(M) = \int_{\mathcal{F}/\mathcal{G}} e^{-S(A)} \mathcal{D} A,
\end{equation}
weighted by the action map $S: \mathcal{F}/\mathcal{G} \to \C$.

Because $\mathcal{F}$ is infinite-dimensional, it is in general a difficult open problem to make sense of integrals of the above type.
However, if $M \cong |X|$ comes with a specified triangulation, we could consider replacing previously considered continuum gauge fields with combinatorial gauge fields assigned to simplices.
This approach is followed in lattice field theories such as lattice quantum chromodynamics.
We are interested in the case where the combinatorial data is sufficiently finite, so that we can define the path integral \ref{eq:pathintegral} as a state sum.
Our aim is to put these sums into the formalism of Section \ref{sec:statesum}.

For this, we start by discussing Dijkgraaf-Witten theory.
We will rediscover special cases of the examples we discussed in Sections \ref{sec:2D} and \ref{sec:3D} in a more physical framework, compare \cite[Sections 3 and 4]{freed2009topological}.
So let us consider a gauge theory on $M = |X|$ for $G$ a finite group.
 In that case, every principal $G$-bundle $P$ has a unique connection and $\mathcal{F}/\mathcal{G} = H^1(M,G)$ is a finite set given by the collection of isomorphism classes of principal $G$-bundles.
 Combinatorially, we can specify a principal $G$-bundle on $M$ by assigning elements $g \in G$ to $1$-simplices in $X$.
 This assignment satisfies a further cocycle condition on $2$-simplices, requiring it to be a simplicial $1$-cochain on $M$ with values in $G$.
More abstractly, we see that this a functor $\mathcal{O}(X) \to BG$.
Indeed, note that since $BG$ is a groupoid, we obtain by the observations in subsection \ref{sec:groupoidify} that this functor is equivalent to a map of simplicial sets $X \to BG$.
This motivates the following definition.

\begin{definition}
\label{def:DWgaugefield}
    A \emph{combinatorial $G$-gauge field} is a functor $A: \mathcal{O}(X) \to BG$.
\end{definition}

Topologically, the space $\Map(X,BG)$ of gauge fields is aspherical and its fundamental groupoud $\Pi_1(\Map(X,BG))$ is equivalent to the groupoid $\mathcal{F}//\mathcal{G}$ of principal bundles over $M$.
 Now let $\omega: BG \to B^n \C^\times$ be a map of simplicial sets.
 This map represents a cohomology class in $H^n(BG; \C^\times)$ known as the twist, which allows one to define the exponentiated action functional of Dijkgraaf-Witten theory as
\[
e^{S(A)} := \int_X A^*(\omega),
\]
where $A^*$ is the pullback along $A: X \to BG$.
In this combinatorial context, the integral is defined by evaluating on the fundamental class.
This allows for a rigorous definition of the partition function 

\begin{definition}
\label{def:partitionfunctionDW}
The \emph{partition function} of $\omega$-twisted Dijkgraaf-Witten theory on the spacetime $X$ is the finite sum
\[
Z(X) = \sum_{A \in [X,BG]} \int_X A^*(\omega) \in \C^\times.
\]
\end{definition}

We will now explain in low dimensions how this is a special case of a state-sum in the sense of Definition \ref{def:statesuminvariant}.

We start in dimension two.
Note that a map $\omega: BG \to B^2 \C^\times$ of simplicial sets gives an explicit group $2$-cocycle on $G$, which we again denote by $\omega$.
We will assume $\omega$ is a unital cocycle in the sense that $\omega(g,1) = \omega(1,g) = 1$. \LS{Not assuming this probably messes up some of the normalizations below}
Define the $\omega$-twisted group algebra $\C^\omega G$ as the complex algebra with generators $x_g$ for every $g \in G$ and relations
\[
x_g x_h = \omega(g,h) x_{gh}.
\]
It is a well-known fact that a unital and counital Frobenius structure is uniquely determined by specifying the counit, which we define to be
\[
\epsilon(x_g) = \delta_{g,e} |G|.
\]
The normalization is chosen in a way that the induced Frobenius algebra is $\Delta$-separable.
Explicitly, the coproduct is given by
\[
\Delta(x_g) = \frac{1}{|G|} \sum_{h \in G} x_g x_{h}^{-1} \otimes x_h.
\]
This Frobenius algebra gives a two-dimensional state-sum model $F$ as explained in Section \ref{sec:2D}.\LS{We could also mention the twisted group algebra already in the 2d section}
We claim that the partition function of a triangulated surface $\Sigma$ gives the state sum invariant $F(\sigma_\Sigma)$ in the sense of \ref{def:statesum}, compare \cite{oritthesis}. \LS{explain this or not necessary?}

We now consider three-dimensional Dijkgraaf-Witten theory.
In this case, we will consider the state-sum data with values in $B\CYCat$ given by $G$-graded vector spaces $\Vect^\omega_G$ with associator $\omega$, see Example \ref{ex:3dDW}.
The state-sum is then the Turaev-Viro invariant based on $\Vect^\omega_G$, which is easily seen to be the partition function of Dijkgraaf-Witten theory. 
In four dimensions, a state-sum has been constructed for spherical fusion $2$-categories \cite{douglasreutter}. \LS{maybe we need to mention this earlier}
Again there exists an $\omega$-twisted $G$-graded spherical fusion $2$-category for $\omega \in H^4(BG; \C^\times)$ and the corresponding state-sum agrees with Dijkgraaf-Witten theory \cite[3.4.2.]{douglasreutter}.

More generally, we expect a state-sum for $n$-dimensional Dijkgraaf-Witten theory to be constructed from a map of simplicial sets $\omega: BG \to B^n \C^\times$ as follows.
Consider the canonical inclusion $B^n \C^\times \to B^{n-1} \Vect \to (n-1)\Cat_\C$ into \textcolor{red}{many adjectives} $\C$-linear $(n-1)$-categories. \LS{What kind of orbifold completion of $B^{n-1} Vect$ should this be? something something cauchy/Karoubi complete? We also need full dualizability of $(n-1)\Cat_\C$, so finite-dim homs, finitely many simples...?}
 The limit of the functor $BG \to (n-1)\Cat_\C$ defines a $\C$-linear $(n-1)$-category with a single object, i.e. a monoidal $(n-2)$-category $\Vect^G_\omega$ \LS{or should we do colimit? Or maybe it does not matter}.
 The resulting category generalizes the $\omega$-twisted group algebra for the case $n = 2$ and the category of $\omega$-twisted $G$-graded vector spaces for the cases $n = 3$ and $n=4$.
 The monoidal structure generalizes the multiplication on $\C^\omega G$ and the monoidal structure on $\omega$-twisted $G$-graded vector spaces respectively.
 We expect to be able to associate state-sum data to this in $(n-1)\Cat_\C$.
We assign $\Vect^G_\omega$ to $1$-simplices, to $2$-simplices we assign the tensor product on this category and to $3$-simplices we assign the associator and we continue. \LS{Is there a nice way to describe all of these together more coherently?}
Note that because of \textcolor{red}{many adjectives}, the tensor product has an adjoint $\mathcal{C} \to \mathcal{C} \boxtimes \mathcal{C}$, which allows us to compose $2$-simplices in different directions.
...

In the rest of the section, we will speculate about the relationship between more general state sums and physics, in particular for noninvertible symmetries.
Indeed, motivated by the example of Dijkgraaf-Witten theory, we are led to conclude that in case the state-sum data is constructed freely from a (higher) group $G$, we can thing of the state-sum values as the result of gauging the (higher form) symmetry $G$.
Our hope is that the setup of this article will allow us to generalize examples of the above type from higher groups $G$ to higher categories, yielding a mathematically rigorous way to gauge noninvertible symmetries.
Note that if we would have used the simplicial set $X$ as a higher groupoid in place of the category $\mathcal{O}(X)$, there would be no hope to see non-invertible gauge fields as every map into a higher category would factor through its maximal subgroupoid.

In the example of Dijkgraaf-Witten theory in a general dimension, we constructed a fusion $(n-2)$-category $C$ and looked at the induced state sum with values in $(n-1)\Cat_\C$.
Moreover, these fusion categories their simple objects labeled by the elements of $G$.
For example, in the case of $2$-dimensional Dijkgraaf-Witten, the group ring has a canonical basis coming from a fusion ring-like structure. \LS{Lukas asked: How to make simple things compatible between different layers?}
For those cases, we would like define a field on $X$ as a consistent assignment of simple objects of $C$ to simplices of $X$.
In other words, we would like to define a \emph{$C$-valued lattice gauge field on $X$} as a functor $\mathcal{O}(X) \to BC$, which is required to land in the simple objects. 
We immediately arrive at a problem: the tensor product of simple objects is only a sum of simple objects in general. 
Therefore simple objects are not a subcategory of $BC$ and so lattice gauge fields are not a subcategory of $\Fun(\mathcal{O}(X), BC)$.
Therefore we will from work with all of $\Fun(\mathcal{O}(X), BC)$, keeping in mind that these should be thought of as `formal $\C$-linear combinations of lattice gauge fields'.
...

An action functional on $X$ in this formulation, should be a functor $S: \Fun(\mathcal{O}(X), BC) \to B^{n-1} \C^\times$.
Here we defined $S$ not only on the simple object-valued fields, so that the fusion product is correctly encoded.
Actually, we will only consider action functionals that are `functorial in spacetime'.
By the universal property of $\mathcal{O}(X)$ this means:

\begin{definition}
    An \emph{action functional for a $C$-lattice gauge theory} is ....
\end{definition}

Explicitly, the data of an action functional gives a map from lattice gauge fields on $\Delta^n$ to $\C$ such that....
On a general $X$, this induces a map $\Fun(\mathcal{O}(X), BC) \to \C$...

Then the path integral has to be a sort of sum....

\textcolor{red}{
Outline of the rest
\begin{enumerate}
    \item cases where one has an action functional
    \item relationship between the action functional $S$ and the state sum $F$ 
    \item $F(\sigma_X)$ is the partition function?
    \item comment on non-invertible gauge theories (Lukas: there might be a problem with formulating classical gauge fields for non-invertible symmetries)
\end{enumerate}}



TODO: read old text and save things that make sense.

\textcolor{red}{old text starts}

\color{red} Mention cochains and lattice field theories. \color{black}

The first task in our construction of non-invertible Dijkgraaf--Witten theory
is to find an appropriate generalisation of the space of fields. A first guess
on how to overcome this problem is to work with higher categories instead of
(higher) groupoids. For example when one replaces the finite group $G$ by an
finite dimensional algebra $A$, it is natural to consider the the linear category
$BA$ as a natural replacement of $BG$. However, the problem with this is
that any map from $M$ to $BA$ will automatically factor through the full
subgroupoid of $BA$. Alternativly, one might try to consider the space or
$\infty$-groupoid constructed by geometrically realising $BA$. The geometric
realisation is a concrete way of constructed the localisation of $BA$ at all
morphisms. A gauge theory based on the geometric realisation $|BA|$ replaces the
non-invertible gauge structure by a higher gauge group $|BA|$. Higher gauge
theoretic versions of Dijkgraaf--Witten theory have been studied
extensively~\cite{some thing}, but are not what we are interested in here.



\color{red}
Explicit description and relation to DW.

I should really read the stuff on higher Segal spaces, again \color{black}