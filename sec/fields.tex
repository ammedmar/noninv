% !TEX root = ../noninv.tex

\section{Categories of fields and actions} 

In traditional Dijkgraaf--Witten theory the fields are principal bundles for a finite
group $G$. These form a groupoid $\Bun_G$ with morphisms given by gauge 
transformations. An alternative, but equivalent, description of the collection of all fields on a manifold $M$ is as the mapping space $\Map(M,BG)$. The connection
between the two description is through the homotopy hypothesis~\cite{} which is
an equivalence between $\infty$-groupoids and topological spaces. Concretely,
the space $\Map(M,BG)$ is aspherical and its fundamental groupoud $\Pi_1 
(\Map(M,BG))$ is equivalent to $\Bun_G(M)$.
This has an explicit description in terms of a triangulation of $X$ of $M$. 

\color{red} Mention cochains and lattice field theories. Note that this actually 
depends on the triangulation \color{black}

The first task in our construction of non-invertible Dijkgraaf--Witten theory
is to find an appropriate generalisation of the space of fields. A first guess
on how to overcome this problem is to work with higher categories instead of
(higher) groupoids. For example when one replaces the finite group $G$ by an 
finite dimensional algebra $A$, it is natural to consider the the linear category
$BA$ as a natural replacement of $BG$. However, the problem with this is 
that any map from $M$ to $BA$ will automatically factor through the full 
subgroupoid of $BA$. Alternativly, one might try to consider the space or 
$\infty$-groupoid constructed by geometrically realising $BA$. The geometric 
realisation is a concrete way of constructed the localisation of $BA$ at all 
morphisms. A gauge theory based on the geometric realisation $|BA|$ replaces the 
non-invertible gauge structure by a higher gauge group $|BA|$. Higher gauge 
theoretic versions of Dijkgraaf--Witten theory have been studied 
extensively~\cite{some thing}, but are not what we are interested in here. 

We propose a different approach to overcome these problems which generalises the
lattice formulation outlined above. The key observation is that as explained in 
Section~... to a triangulation together with a branching structure one can naturally associate an $d$-category $\Or(X)$. Restricting to the $n$-skeleton  
of $X$ gives a $n$-category $\Or_n(X)$. 
\begin{definition}
Let $\mathcal{C}$ be a fusion $n$-category and $X$ a triangulation together with 
the choice of branching structure a triangulation of an oriented $d$-dimensional manifold.
The linear $n+1$-category $\mathcal{F}_\mathcal{C}(X)$ of 
\emph{$\mathcal{C}$-lattice gauge fields on $X$} is the category of functors from $\Or_{n+1}(X)$ to the $n+1$-category $B\mathcal{C}$, which is the delooping of $\mathcal{C}$. 
\end{definition}

\color{red} 
Explicit description and relation to DW.

I should really read the stuff on higher Segal spaces, again \color{black}