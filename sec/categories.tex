% !TEX root = ../noninv.tex

\section{Triangulated oriented manifolds as higher categories}

The purpose of this section is to describe how to think of a triangulated oriented $n$-manifold as an $n$-category.

\subsection{Simplicial sets}

We set our basic notation and refer the reader to \cite{friedman2021simplicial} for a leisurely exposition of the theory of simplicial sets.

The \textit{simplex category} is denoted by $\simplex$ and the category of \textit{simplicial sets} $\Fun(\simplex^\op, \Set)$ as $\sSet$.
As usual, for any object $[n]$ of $\simplex$ the \textit{standard $n$-simplex} is denoted by $\simplex^n$, and we have
\[
X_n \defeq X[n] \cong \sSet(\simplex^n, X).
\]
The category $\simplex \downarrow X$ has objects given by morphisms $\simplex^n \to X$ in $\sSet$ and morphisms given by commuting triangles
\[
\begin{tikzcd}[column sep=-5pt, row sep=small]
	\simplex^{n_1} \arrow[dr] \arrow[rr] & & \simplex^{n_2}  \arrow[dl] \\
	& X &
\end{tikzcd}
\]
therein.
Denoting the canonical forgetful functor $\simplex \downarrow X \to \sSet$ by $\pi$ we have
\[
X \cong \colim_{\simplex \downarrow X} \pi.
\]

\subsection{Geometric realization}

The geometric $n$-simplex is denoted by $\gsimplex^n$.
Together with the usual inclusions these define a functor
\[
\begin{split}
	\bars{-} \colon \simplex \, &\to \CW \\
	[n] &\mapsto \, \gsimplex^n.
\end{split}
\]
Omitting the canonical projection $\simplex \downarrow X \to \sSet$, the \textit{geometric realization} of a simplicial set $X$ is defined by
\[
\bars{X} \defeq \colim_{\simplex \downarrow X} \,\bars{-}.
\]

\subsection{Simplicial complexes}

A simplicial set is said to be a \textit{simplicial complex} if it has the following property: The $n+1$ vertices of any non-degenerate $n$-simplex $x$ are distinct and this set completely determine $x$ among non-degenerate $n$-simplices.
We remark that this definition is equivalent to the more traditional notion of an (abstract) simplicial complex equipped with a branching structure, that is, a compatible choice of total order on the vertices of every simplex.

\subsection{Triangulated manifolds}

Let $M$ be an $n$-manifold, which for us is always smooth.
A \textit{triangulation} of $M$ is a homeomorphism $M \cong \bars{X}$, where $X$ is a simplicial complex, such that each of the \textit{characteristic maps}
\[
\gsimplex^k \to M
\]
extends to a smooth map in a neighborhood of $\gsimplex^k$.

%Since $M$ is a manifold, $\bars{X}$ can be computed using a subcategory of $\simplex \downarrow X$ consisting of only top dimensional simplices and codimension 1 faces.

\subsection{Orientations}

Each geometric simplex is equipped with a canonical orientation induced from Euclidean space.
If $M$ is an oriented $n$-manifold then each characteristic map $\gsimplex^n \to M$ is either orientation preserving or reversing.

\subsection{Higher categories}

We use the recursive definition of $n$-category: a $0$-category is a set and an \textit{$n$-category} is a category enriched over $(n-1)$-categories.
An \textit{\mbox{$\omega$-category}} is defined as the limiting object when $n$ goes to $\infty$.
We denote the category of $\omega$-categories as $\omega\Cat$.

\subsection{Street categories}

The free $n$-category generated by the $n$-simplex is denoted by $\oriental_n$.
These were introduced by Street in \cite{street1987orientals}.
The precise definition is somewhat arduous to present.
Intuitively, this $n$-category has a $k$-morphism for each non-degenerate $k$-simplex of $\simplex^n$ as well as all possible compositions of these.
Consult Figure ?? for a pictorial depiction of low dimensional examples.
We denote by $\oriental_n^\op$ the same category but with the top dimensional morphism going in the opposite direction.

\subsection{Categorical realization}

The Street categories, together with natural functors between them, define a functor
\[
\begin{split}
	\cO \colon \simplex \, &\to \wCat \\
	[n] &\mapsto \, \cO_n.
\end{split}
\]
Omitting the canonical projection $\simplex \downarrow X \to \sSet$, the \textit{categorical realization} of a simplicial set $X$ is defined by
\[
\oriental(X) = \colim_{\simplex \downarrow X} \oriental.
\]

\subsection{Triangulated manifolds as higher categories}

The $n$-category of a triangulated $n$-manifold $(M, X)$ is $\oriental(X)$.
When $M$ is oriented we modify this definition as follows.

