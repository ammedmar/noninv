% !TEX root = ../noninv.tex

\appendix
\section{Higher categories}\label{s:categories}

\subsection{$\omega$-categories}

We will consider $n$-\textit{categories} recursively defined as categories enriched in $(n-1)$-categories.
We will take the one-sorted viewpoint identifying an $n$-category $\cC$ with a set $\Mor\cC$ together with, for $i = 0,\dots n-1$, \textit{source} and \textit{target functions}
\[
s_i, t_i \colon \Mor\cC \to \Mor\cC
\]
and \textit{partial compositions}
\[
\circ_i \colon \Mor\cC \times \Mor\cC \to \Mor\cC
\]
with $a \circ_i b$ defined when $s_i a = t_i b$.
For a complete list of relations satisfied by these please consult \cite[Definition 2.1]{steiner2004omega}.
An $n$-category is naturally a $(n+1)$-category, and we define $\omega$-category as the limiting notion.

We now will describe $\omega$-categories defined freely from combinatorial data, generalizing the construction of $1$-categories from directed graphs.

\subsection{Pasting schemes}

Let $C$ be a chain complex with a basis $B$.
For any $c \in C$, writing $\bd\, c$ in the basis, we define $\bd^\varepsilon c$ for $\varepsilon \in \{+,-\}$ by the equation
\[
\bd\, c = \bd^+c - \bd^-c.
\]
For any $\varepsilon \in \{+,-\}$ and basis element $b$ define its \textit{atom} recursively by
\[
\angles{b}^\varepsilon_k =
\begin{cases}
	0 & \bars{b} > k, \\
	b & \bars{b} = k, \\
	\bd^\varepsilon \angles{b}^\varepsilon_{k+1} & \bars{b} < k.
\end{cases}
\]
For two basis elements of degree greater than $k$ we write $a <_k b$ if there is a basis element with a nonzero coefficient in both $\angles{a}^+_k$ and $\angles{b}^-_k$.

\begin{definition*}
	A \textit{pasting scheme} is a chain complex with a basis $B$ such that:
	\begin{enumerate}
		\item\label{i:loop-free} For each $k$, the relation $<_k$ makes $B$ into a poset.
		\item\label{i:unital} For each $b \in B$ and $\varepsilon \in \{+,-\}$, the element $\angles{b}^\varepsilon_0$ is equal to a basis element.
	\end{enumerate}
	We we will refer to these as the \textit{loop-free} and \textit{unital} conditions respectively.
\end{definition*}

We will mostly concern ourselves with the notion of pasting scheme as defined above.
But, for completeness, we will review in the next two subsections their freely generated $\omega$-categories.

\subsection{Free $\omega$-categories}

The starting point of Steiner's construction is a form of Dold--Kan correspondence appearing in work by R. Brown and P. J. Higgins \cite{brown1981cubes}.
They show that (non-negatively graded) chain complexes are equivalent to $\omega$-category objects in the category of abelian groups using a functor we now described on objects.
Given a chain complex $C$, we consider all elements
\[
c = (c_0^-,c_0^+,c_1^-,c_1^+,\dots)
\]
in the infinite product of $C$ with itself satisfying the following conditions for all $k \in \N$ and $\varepsilon \in \{-,+\}$:
\begin{enumerate}
	\item $c_k^\varepsilon \in C_k$,
	\item $c_k^\varepsilon \neq 0$ except for finitely many $k$,
	\item $\bd c_{k+1}^\varepsilon = c_k^+ - c_k^-$.
\end{enumerate}
We now describe for each $k \in \N$, its source and target functions
\begin{align*}
	&s_k(c) = (c_0^-,c_0^+,\dots,c_{k-1}^-,c_{k-1}^+,c_k^-,c_k^-,0,0,\dots),\\
	&t_k(c) = (c_0^-,c_0^+,\dots,c_{k-1}^-,c_{k-1}^+,c_k^+,c_k^+,0,0,\dots),
\end{align*}
and partial composition
\[
c \circ_k c' = c+c'-c'',
\]
where $t_k(c) = s_k(c') = c''$.

Steiner defines the $\omega$-category freely generated by a pasting scheme by considering the $\omega$-category generated by the atoms of the pasting scheme inside this abelian $\omega$-category.

\subsection{Street's orientals}\label{ss:orientals}

Let $\gchains(\gsimplex^n)$ be the usual chain complex of the standard $n$-simplex.
Explicitly, its basis of degree $k$ elements consists of tuples $[v_0,\dots,v_k]$ of distinct integers in $\{0,\dots,n\}$, with boundary defined by
\[
\bd\, [v_0,\dots,v_k] = \sum_{i=0}^k (-1)^i [v_0,\dots,\widehat{v}_i,\dots,v_k].
\]
To verify that this based chain complex satisfies the loop-free and unital conditions, please refer to \cite[Example 3.8]{steiner2004omega}.
Examples of atoms associated with this pasting scheme are provided in \cref{f:street}.
As demonstrated by Steiner in the same reference, this pasting scheme defines a free $\omega$-category that is isomorphic to the Street oriental $\cO_n$ \cite{street1987orientals}.

\begin{figure}
	\centering
	\begin{small}
	\begin{verbatim}
		[0,] | [1,]
		---------------
		[0,] | [2,]
		[0,2] | [0,1] + [1,2]
		---------------
		[0,] | [3,]
		[0,3] | [0,1] + [1,2] + [2,3]
		[0,2,3] + [0,1,2] | [0,1,3] + [1,2,3]
		---------------
		[0,] | [4,]
		[0,4] | [0,1] + [1,2] + [2,3] + [3,4]
		[0,3,4] + [0,2,3] + [0,1,2] | [0,1,4] + [1,2,4] + [2,3,4]
		[0,2,3,4] + [0,1,2,4] | [0,1,2,3] + [0,1,3,4] + [1,2,3,4]
	\end{verbatim}
\end{small}
	\caption{The (non-trivial part of the) atom of the top dimensional basis element in $\gchains(\gsimplex^n)$ for $n \in \{1,2,3,4\}$. (Computations done using \href{https://github.com/ammedmar/wcat}{\texttt{wcat}}.)}
	\label{f:street}
\end{figure}