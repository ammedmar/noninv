% !TEX root = ../noninv.tex

\section{Triangulated oriented manifolds as higher categories}

The purpose of this section is to describe how to think of a triangulated oriented $n$-manifold as an $n$-category.

\subsection{$\simplex$-sets}

We set our basic notation and refer the reader to \cite{friedman2021simplicial} for a leisurely exposition of the theory of simplicial and $\simplex$-sets.
For us, the latter will play a more important role.

The \textit{delta category} $\simplex$ has objects the posets $[n] = \set{0 < \dots < n}$ for $n \in \N$ and its morphisms are \textit{strictly} order preserving maps.
Diagrammatically,
\[
\begin{tikzcd}
	{[0]} \arrow[r,shift left=3pt, "\delta_0"] \arrow[r,shift right=3pt, "\delta_1"'] &
	{[1]} \arrow[r,shift left=6pt, "\delta_0"] \arrow[r, "\delta_1"' description] \arrow[r,shift right=6pt, "\delta_2"'] &
	\dotsb
	\arrow[r,shift left=5pt, "\delta_0"] \arrow[r,shift right=5pt, "\delta_n"', "\scalebox{0.6}{$\vdots$}"] &
	{[n]}
	\arrow[r,shift left=5pt, "\delta_0"] \arrow[r,shift right=5pt, "\delta_{n+1}"', "\scalebox{0.6}{$\vdots$}"] &
	\dotsb \
\end{tikzcd}
\]
where $\delta_i$ is the $i^\th$ \textit{coface map}.
The \textit{$n$-truncation} of the delta category is its full subcategory with set of objects $\set{[k] \mid k \leq n}$.

The category of \textit{$\simplex$-sets} and \textit{$\simplex$-maps} $\Fun(\simplex^\op, \Set)$ is denoted by $\dSet$, and the \textit{standard $n$-simplex} $\dSet(-, [n])$ by $\simplex^n$.

For any $\simplex$-set $X$ the elements in $X_n \defeq X[n]$ are referred to as its $n$-simplices and there is a bijection
\[
\begin{tikzcd}[row sep=-2pt]
	X_n \arrow[r,"\cong"] & \dSet(\simplex^n, X) \\
	x \arrow[r,mapsto] & \iota_x \colon \simplex^n \to X,
\end{tikzcd}
\]
where $\iota_x$ is referred to as the \textit{characteristic map} of $x$.

For a $\simplex$-set $X$, the category $\simplex \downarrow X$ has objects given by maps $\simplex^n \to X$ and morphisms given by commuting triangles
\[
\begin{tikzcd}[column sep=-5pt, row sep=small]
	\simplex^{n_1} \arrow[dr] \arrow[rr] & & \simplex^{n_2}  \arrow[dl] \\
	& X &
\end{tikzcd}
\]
of $\simplex$-maps.
Omitting the canonical functor $\simplex \downarrow X \to \dSet$ forgetting $X$, we have that $X \cong \colim_{\simplex \downarrow X}$.

\subsection{Geometric realization}

The geometric $n$-simplex is denoted by $\gsimplex^n$.
Together with the usual inclusions these define a functor
\[
\begin{split}
	\bars{-} \colon \simplex \, &\to \CW \\
	[n] &\mapsto \, \gsimplex^n.
\end{split}
\]
Omitting the canonical functor $\simplex \downarrow X \to \dSet$, the \textit{geometric realization} of a $\simplex$-set $X$ is defined by
\[
\bars{X} \defeq \colim_{\simplex \downarrow X} \,\bars{-}.
\]

\subsection{Higher categories}

We use the recursive definition of $n$-category: a $0$-category is a set and an \textit{$n$-category} is a category enriched over $(n-1)$-categories.
An \textit{\mbox{$\omega$-category}} is defined as the limiting object when $n$ goes to $\infty$.
We denote the category of $\omega$-categories as $\omega\Cat$.

\subsection{Street categories}

The free $n$-category generated by the $n$-simplex is denoted by $\oriental_n$.
These were introduced by Street in \cite{street1987orientals}.
The precise definition is somewhat arduous to present.
Intuitively, this $n$-category has a $k$-morphism for each $k$-simplex in $\simplex^n$ as well as all possible compositions of these morphisms.
Consult Figure ?? for a depiction of low dimensional examples.
We denote by $\oriental_n^\op$ the same category but with the top dimensional morphism going in the opposite direction.

\subsection{Categorical realization}

The Street categories, together with natural inclusion between them, define a functor
\[
\begin{split}
	\cO \colon \simplex \, &\to \wCat \\
	[n] &\mapsto \, \cO_n.
\end{split}
\]
Omitting the canonical functor $\simplex \downarrow X \to \dSet$, the \textit{categorical realization} of a simplicial set $X$ is defined by
\[
\oriental(X) = \colim_{\simplex \downarrow X} \oriental.
\]

\subsection{Simplicial complexes}

A simplicial set is said to be a \textit{simplicial complex} if it has the following property: The set of $n+1$ vertices of any non-degenerate $n$-simplex $x$ are distinct and completely determines $x$ among non-degenerate $n$-simplices.
We remark that this definition is equivalent to the more traditional notion of an (abstract) simplicial complex equipped with a branching structure, that is, a compatible choice of total order on the vertices of every simplex.\anibal{Change this from sset to dset.}

\subsection{Triangulated manifolds}

Let $M$ be an $n$-manifold, which for us is always smooth.
A \textit{triangulation} of $M$ is a homeomorphism $M \cong \bars{X}$, where $X$ is a simplicial complex, with the property that each map $\gsimplex^k \to M$ induced by a \textit{characteristic map} extends to a smooth map in a neighborhood of $\gsimplex^k$.

\subsection{Colimits over triangulated manifolds}

Let $M \cong \bars{X}$ be a triangulated $n$-manifold.
Since any simplex is contained in a non-degenerate $n$-simplex, and any two such share at most a single $(n-1)$-simplex, we have that
\[
X \cong \colim_{\simplex \downarrow X}
  \cong \textstyle \colim \left(\bigsqcup_{X_{n-1}^\nd} \Delta^{n-1} \rightrightarrows \bigsqcup_{X_n^\nd} \Delta^n \right).
\]

\subsection{Orientations}

Each geometric simplex is equipped with a canonical orientation induced from Euclidean space.
If $M \cong \bars{X}$ is an oriented $n$-manifold, then each $\gsimplex^n \to M$ induced by a characteristic map is either orientation preserving or reversing.
We write $X_{n,+}^\nd$ and $X_{n,-}^\nd$ for the associated partition of $X_n^\nd$.

\subsection{Oriented categorical realization}

Let $M \cong \bars{X}$ be a triangulated $n$-manifold with a chosen orientation $\theta$.
We define its \textit{oriented categorical realization} as
\[
\oriental_\theta(X) \defeq
\textstyle \colim \left(\bigsqcup_{X_{n-1}^\nd} \oriental_{n-1} \rightrightarrows \bigsqcup_{X_{n,+}^\nd} \oriental_n \sqcup \bigsqcup_{X_{n,-}^\nd} \oriental_n^\op \right).
\]
Figure ...