% !TEX root = ../noninv.tex

\section{Triangulated oriented manifolds as higher categories}

The purpose of this section is to describe how to think of triangulated oriented $n$-dimensional manifold an $n$-categories.

\subsection{Cones and colimits}

We recall some basic notions from category theory and refer the reader to any standard textbook on the subject for further context and details.

Let us consider a small category $\cJ$ and a category $\cC$.
A \textit{diagram} in $\cC$ of shape $\cJ$ is another name for a functor from $\cJ$ to $\cC$.
A \textit{cone} under a diagram $F \colon \cJ \to \cC$ is an object $c$ of $\cC$ together with a natural transformation from $F$ to the constant functor $c$.
The \textit{colimit} of a diagram $F \colon \cJ \to \cC$, denoted $\colim_\cJ F$, is the cone under $F$ with the following universal property: Given any cone $c$ under $F$, there is a unique morphism from $\colim_\cJ F$ to $c$ making the following diagram commute
\[
\begin{tikzcd}[column sep=-5pt]
	F(j) \arrow[rr] \arrow[dr,in=135, out=-45] \arrow[ddr,in=160,out=-90] & &
	F(j') \arrow[dl,in=45, out=-135] \arrow[ddl,in=20,out=-90] \\
	& \colim_\cJ F \arrow[dashed,d]& \\
	& c &
\end{tikzcd}
\]
for each morphism $j \to j'$ in $\cJ$.
One refers to a category $\cC$ for which this universal cone exists for each $\cJ$ as \textit{cocomplete}.

\subsection{$\simplex$-sets}

We set our basic notation and refer the reader to \cite{friedman2021simplicial} for a leisurely exposition of the theory of simplicial and $\simplex$-sets.
For us, the latter will play a more important role.

The \textit{delta category} $\simplex$ has objects the posets $[n] = \set{0 < \dots < n}$ for $n \in \N$ and its morphisms are \textit{strictly} order preserving maps.
Diagrammatically,
\[
\begin{tikzcd}
	{[0]} \arrow[r,shift left=3pt, "\delta_0"] \arrow[r,shift right=3pt, "\delta_1"'] &
	{[1]} \arrow[r,shift left=6pt, "\delta_0"] \arrow[r, "\delta_1"' description] \arrow[r,shift right=6pt, "\delta_2"'] &
	\dotsb
	\arrow[r,shift left=5pt, "\delta_0"] \arrow[r,shift right=5pt, "\delta_n"', "\scalebox{0.6}{$\vdots$}"] &
	{[n]}
	\arrow[r,shift left=5pt, "\delta_0"] \arrow[r,shift right=5pt, "\delta_{n+1}"', "\scalebox{0.6}{$\vdots$}"] &
	\dotsb \
\end{tikzcd}
\]
where $\delta_i$ is the $i^\th$ \textit{coface map}.

The category of \textit{$\simplex$-sets} and \textit{$\simplex$-maps} $\Fun(\simplex^\op, \Set)$ is denoted by $\dSet$, and the \textit{standard $n$-simplex} $\dSet(-, [n])$ by $\simplex^n$.

For a $\simplex$-set $X$ the elements in $X_n \defeq X[n]$ are referred to as its \textit{$n$-simplices}, and $\face_i \defeq X \delta_i$ is the \textit{$i^\th$-face map}.

For any $n$-simplex $x$ there is a unique $\simplex$-map $\iota_x \colon \simplex^n \to X$, its \textit{characteristic map}.
Furthermore, any $\simplex$-map $\simplex^n \to X$ is the characteristic map of an $n$-simplex.
These maps can be considered to be the objects of a category $\deltaover{X}$ whose morphisms are commuting triangle
\[
\begin{tikzcd}[column sep=-2pt, row sep=small]
	\simplex^{n_1} \arrow[dr] \arrow[rr] & & \simplex^{n_2}  \arrow[dl] \\
	& X &
\end{tikzcd}
\]
of $\simplex$-maps.
Denoting the canonical functor $\deltaover{X} \to \simplex$ by $\pi$ we have
\[
X \cong \colim_{\deltaover{X}} \pi.
\]

\subsection{Geometric realization}

The geometric $n$-simplex is denoted by $\gsimplex^n$.
Together with their usual face inclusions these define a functor
\[
\begin{split}
	\bars{-} \colon \simplex \, &\to \Top \\
	[n] &\mapsto \, \gsimplex^n
\end{split}
\]
and the \textit{geometric realization} of a $\simplex$-set $X$ is
\[
\bars{X} \defeq \colim_{\deltaover{X}}(\bars{-} \circ \pi).
\]

\subsection{Truncations}

The \textit{$n$-truncation} of the delta category, denoted $\simplex_n$, is the full subcategory of $\simplex$ with set of objects $\set[\big]{[m] \mid m \leq n}$.
We say that a $\simplex$-set is \textit{$n$-dimensional} if $X_{n+k} = \emptyset$ for each $k \in \N$.
The full subcategory of $n$-dimensional $\simplex$-sets is naturally equivalent to $\Fun(\simplex^\op_n, \Set)$.

\subsection{Simplicial complexes}

A $\simplex$-set is said to be a \textit{simplicial complex} if it has the following property: The set of $n+1$ vertices of any $n$-simplex $x$ are distinct and completely determines $x$.
We remark that this set of vertices is totally ordered, since face maps are.

\subsection{Triangulated manifolds}

Let $M$ be an $n$-manifold, which for us is always smooth.
A \textit{triangulation} of $M$ is a homeomorphism $M \cong \bars{X}$, where $X$ is an ($n$-dimensional) simplicial complex, with the property that each map $\gsimplex^m \to M$ induced by a characteristic map extends to a smooth map in a neighbourhood of $\gsimplex^m$.

\subsection{Orientations}

Each geometric simplex is equipped with a canonical orientation induced from Euclidean space.
If $M \cong \bars{X}$ is an oriented $n$-manifold, then each $\gsimplex^n \to M$ induced by a characteristic map either preserves or reverses orientation.
Let us consider two copies of the standard $n$-simplex, denoted $\simplex^n_+$ and $\simplex^n_-$, and define the category $\widedeltaover{X}$ to have objects characteristic maps $\simplex^m \to X$ if $m < n$, and, for each characteristic map $\iota_x \colon \simplex^n \to X$, an object $\iota_x \colon \simplex^n_+ \to X$ or $\iota_x \colon \simplex^n_- \to X$ depending on if the smooth map induced by $\iota_x$ preserves or reverses orientation.

\subsection{Oriented truncations}

The category $\widetilde\simplex_n$ is the push-out
\[
\begin{tikzcd}
	\simplex_{n-1} \rar \arrow[d,shift right=3pt] & \simplex_n \arrow[d,shift right=4pt] \phantom{.}\\  \simplex_n \rar & \widetilde\simplex_n.
\end{tikzcd}
\]
Explicitly, it has an object $[m]$ for each $m < n$ and two objects $[n]^+$ and $[n]^-$, both isomorphic to $[n]$.

If $M \cong \bars{X}$ is an oriented $n$-manifold we denote the canonical functor from $\widedeltaover{X}$ to $\widetilde\simplex_n$ by $\widetilde\pi$.

\subsection{Higher categories}

We use the following recursive definition of $n$-category: a $0$-category is a set and an \textit{$n$-category} is a category enriched over $(n-1)$-categories.
We denote the category of $n$-categories by $\nCat$.

\subsection{Street's orientals}

The free $n$-category generated by the $n$-simplex is denoted by $\oriental_n$.
These were introduced by Street in \cite{street1987orientals}.
The precise definition is somewhat arduous to present.
Intuitively, this $n$-category has a $k$-morphism for each $k$-simplex in $\simplex^n$ as well as all possible compositions of these morphisms.
Consult Figure ?? for a depiction of low dimensional examples.

\subsection{Categorical realization}

The Street categories, together with certain natural inclusion between them, define a functor
\[
\begin{split}
	\cO \colon \simplex_n \, &\to \nCat \\
	[m] &\mapsto \, \cO_m
\end{split}
\]
and the \textit{categorical realization} of an $n$-dimensional simplicial set $X$ is defined by
\[
\oriental(X) = \colim_{\deltaover{X}} (\oriental \circ \pi).
\]

\subsection{Oriented categorical realization}

We denote by $\oriental_n^\op$ the same category as $\oriental_n$ but with the top dimensional morphism going in the opposite direction.
We obtain a functor $\widetilde{\oriental} \colon \widetilde{\simplex}_n \to \nCat$ by sending $[n]^+$ and $[n]^-$ to $\oriental_n$ and $\oriental_n^\op$ respectively.

If $M \cong \bars{X}$ is a triangulated oriented $n$-manifold, we define its categorical realization as
\[
\widetilde\oriental(X) = \colim_{\widedeltaover{X}} (\widetilde\oriental \circ \widetilde\pi).
\]